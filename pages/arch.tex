\xsection{Architecture Changes}{5}
This revision of \textit{Malphas} has subtely altered the architecture of the project and its arrangement of submodules.
The most significant change is the introduction of a \textbf{Proxt Server}. The reason for this new module was the introduction
of authentication via \textbf{OAuth}, and was further motivated by the lack of motivation of the developers to implement said
authentication facilities in \verb|C++|, which was and still is the preferred backend language of choice. Thus, a strategy
was devised to outsource this daunting responsibility onto a lightweight proxy. The proxy itself is written in \verb|Kotlin|
and is powered by \verb|Ktor|, a minimal and lightweight web framework designed for implementing microservices. \\[\baselineskip]
The proxy implements two additional routes, namely \verb|/login| and \verb|/callback|, which are used for \textbf{OAuth}:
The \verb|/login| route redirects the user to Google's authentication frontend\footnote{
	Google is the OAuth provider we decided to go with for this project.
} which, upon completion, redirects the user back to \verb|/callback|. This is where the access token is acquired and
the user is sent an authentication cookie containing the aforementioned token, their user ID, as well as an \textbf{OAuth}
``state''. The user is subsequently redirected back to the \textit{Malphas} frontend.
Besides the two aforementioned routes, the proxy passess all requests onto the original backend at a configurable \verb|URL|,
where they are dealt with accordingly. However, the proxy will only forward requests containing the authentication cookie
with a valid access token, unless the route is explicitly declared public in the proxy config. \\[\baselineskip]
Since the entirety of the authentication process has been delegated away to another service, parts of the frontend as
well as the original backend which were previously responsible for this process could be removed entirely. As such,
the authentication endpoints were removed from the original backend. A database migration was performed to remove any
tables and remaining columns that previously dealt with authentication. Since Google relies on string values for identifying
their users, the type of the \verb|ID| field was also changed to \verb|TEXT|. 
